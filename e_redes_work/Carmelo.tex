% Options for packages loaded elsewhere
\PassOptionsToPackage{unicode}{hyperref}
\PassOptionsToPackage{hyphens}{url}
%
\documentclass[
  14pt,
]{article}
\usepackage{amsmath,amssymb}
\usepackage{iftex}
\ifPDFTeX
  \usepackage[T1]{fontenc}
  \usepackage[utf8]{inputenc}
  \usepackage{textcomp} % provide euro and other symbols
\else % if luatex or xetex
  \usepackage{unicode-math} % this also loads fontspec
  \defaultfontfeatures{Scale=MatchLowercase}
  \defaultfontfeatures[\rmfamily]{Ligatures=TeX,Scale=1}
\fi
\usepackage{lmodern}
\ifPDFTeX\else
  % xetex/luatex font selection
\fi
% Use upquote if available, for straight quotes in verbatim environments
\IfFileExists{upquote.sty}{\usepackage{upquote}}{}
\IfFileExists{microtype.sty}{% use microtype if available
  \usepackage[]{microtype}
  \UseMicrotypeSet[protrusion]{basicmath} % disable protrusion for tt fonts
}{}
\makeatletter
\@ifundefined{KOMAClassName}{% if non-KOMA class
  \IfFileExists{parskip.sty}{%
    \usepackage{parskip}
  }{% else
    \setlength{\parindent}{0pt}
    \setlength{\parskip}{6pt plus 2pt minus 1pt}}
}{% if KOMA class
  \KOMAoptions{parskip=half}}
\makeatother
\usepackage{xcolor}
\usepackage[margin=1in]{geometry}
\usepackage{color}
\usepackage{fancyvrb}
\newcommand{\VerbBar}{|}
\newcommand{\VERB}{\Verb[commandchars=\\\{\}]}
\DefineVerbatimEnvironment{Highlighting}{Verbatim}{commandchars=\\\{\}}
% Add ',fontsize=\small' for more characters per line
\usepackage{framed}
\definecolor{shadecolor}{RGB}{248,248,248}
\newenvironment{Shaded}{\begin{snugshade}}{\end{snugshade}}
\newcommand{\AlertTok}[1]{\textcolor[rgb]{0.94,0.16,0.16}{#1}}
\newcommand{\AnnotationTok}[1]{\textcolor[rgb]{0.56,0.35,0.01}{\textbf{\textit{#1}}}}
\newcommand{\AttributeTok}[1]{\textcolor[rgb]{0.13,0.29,0.53}{#1}}
\newcommand{\BaseNTok}[1]{\textcolor[rgb]{0.00,0.00,0.81}{#1}}
\newcommand{\BuiltInTok}[1]{#1}
\newcommand{\CharTok}[1]{\textcolor[rgb]{0.31,0.60,0.02}{#1}}
\newcommand{\CommentTok}[1]{\textcolor[rgb]{0.56,0.35,0.01}{\textit{#1}}}
\newcommand{\CommentVarTok}[1]{\textcolor[rgb]{0.56,0.35,0.01}{\textbf{\textit{#1}}}}
\newcommand{\ConstantTok}[1]{\textcolor[rgb]{0.56,0.35,0.01}{#1}}
\newcommand{\ControlFlowTok}[1]{\textcolor[rgb]{0.13,0.29,0.53}{\textbf{#1}}}
\newcommand{\DataTypeTok}[1]{\textcolor[rgb]{0.13,0.29,0.53}{#1}}
\newcommand{\DecValTok}[1]{\textcolor[rgb]{0.00,0.00,0.81}{#1}}
\newcommand{\DocumentationTok}[1]{\textcolor[rgb]{0.56,0.35,0.01}{\textbf{\textit{#1}}}}
\newcommand{\ErrorTok}[1]{\textcolor[rgb]{0.64,0.00,0.00}{\textbf{#1}}}
\newcommand{\ExtensionTok}[1]{#1}
\newcommand{\FloatTok}[1]{\textcolor[rgb]{0.00,0.00,0.81}{#1}}
\newcommand{\FunctionTok}[1]{\textcolor[rgb]{0.13,0.29,0.53}{\textbf{#1}}}
\newcommand{\ImportTok}[1]{#1}
\newcommand{\InformationTok}[1]{\textcolor[rgb]{0.56,0.35,0.01}{\textbf{\textit{#1}}}}
\newcommand{\KeywordTok}[1]{\textcolor[rgb]{0.13,0.29,0.53}{\textbf{#1}}}
\newcommand{\NormalTok}[1]{#1}
\newcommand{\OperatorTok}[1]{\textcolor[rgb]{0.81,0.36,0.00}{\textbf{#1}}}
\newcommand{\OtherTok}[1]{\textcolor[rgb]{0.56,0.35,0.01}{#1}}
\newcommand{\PreprocessorTok}[1]{\textcolor[rgb]{0.56,0.35,0.01}{\textit{#1}}}
\newcommand{\RegionMarkerTok}[1]{#1}
\newcommand{\SpecialCharTok}[1]{\textcolor[rgb]{0.81,0.36,0.00}{\textbf{#1}}}
\newcommand{\SpecialStringTok}[1]{\textcolor[rgb]{0.31,0.60,0.02}{#1}}
\newcommand{\StringTok}[1]{\textcolor[rgb]{0.31,0.60,0.02}{#1}}
\newcommand{\VariableTok}[1]{\textcolor[rgb]{0.00,0.00,0.00}{#1}}
\newcommand{\VerbatimStringTok}[1]{\textcolor[rgb]{0.31,0.60,0.02}{#1}}
\newcommand{\WarningTok}[1]{\textcolor[rgb]{0.56,0.35,0.01}{\textbf{\textit{#1}}}}
\usepackage{graphicx}
\makeatletter
\def\maxwidth{\ifdim\Gin@nat@width>\linewidth\linewidth\else\Gin@nat@width\fi}
\def\maxheight{\ifdim\Gin@nat@height>\textheight\textheight\else\Gin@nat@height\fi}
\makeatother
% Scale images if necessary, so that they will not overflow the page
% margins by default, and it is still possible to overwrite the defaults
% using explicit options in \includegraphics[width, height, ...]{}
\setkeys{Gin}{width=\maxwidth,height=\maxheight,keepaspectratio}
% Set default figure placement to htbp
\makeatletter
\def\fps@figure{htbp}
\makeatother
\setlength{\emergencystretch}{3em} % prevent overfull lines
\providecommand{\tightlist}{%
  \setlength{\itemsep}{0pt}\setlength{\parskip}{0pt}}
\setcounter{secnumdepth}{-\maxdimen} % remove section numbering
\ifLuaTeX
  \usepackage{selnolig}  % disable illegal ligatures
\fi
\usepackage{bookmark}
\IfFileExists{xurl.sty}{\usepackage{xurl}}{} % add URL line breaks if available
\urlstyle{same}
\hypersetup{
  pdftitle={Resolução do Teste},
  pdfauthor={Thomaz Rodrigues Lima a90985},
  hidelinks,
  pdfcreator={LaTeX via pandoc}}

\title{\textbf{Resolução do Teste}}
\author{Thomaz Rodrigues Lima a90985}
\date{}

\begin{document}
\maketitle

13-16, meus dados

\begin{Shaded}
\begin{Highlighting}[]
\NormalTok{VRSA}\OtherTok{\textless{}{-}}\NormalTok{Dados\_Faro }\SpecialCharTok{|\textgreater{}}
  \FunctionTok{filter}\NormalTok{(Concelho }\SpecialCharTok{==} \StringTok{"Vila Real de Santo António"}\NormalTok{) }\SpecialCharTok{|\textgreater{}}
  \FunctionTok{select}\NormalTok{(}\SpecialCharTok{{-}}\NormalTok{Concelho)}

\NormalTok{Bispo}\OtherTok{\textless{}{-}}\NormalTok{Dados\_Faro }\SpecialCharTok{|\textgreater{}}
  \FunctionTok{filter}\NormalTok{(Concelho }\SpecialCharTok{==} \StringTok{"Vila do Bispo"}\NormalTok{) }\SpecialCharTok{|\textgreater{}}
  \FunctionTok{select}\NormalTok{(}\SpecialCharTok{{-}}\NormalTok{Concelho)}

\NormalTok{Tavira}\OtherTok{\textless{}{-}}\NormalTok{Dados\_Faro }\SpecialCharTok{|\textgreater{}}
  \FunctionTok{filter}\NormalTok{(Concelho }\SpecialCharTok{==} \StringTok{"Tavira"}\NormalTok{) }\SpecialCharTok{|\textgreater{}}
  \FunctionTok{select}\NormalTok{(}\SpecialCharTok{{-}}\NormalTok{Concelho)}

\NormalTok{Silves}\OtherTok{\textless{}{-}}\NormalTok{Dados\_Faro }\SpecialCharTok{|\textgreater{}}
  \FunctionTok{filter}\NormalTok{(Concelho }\SpecialCharTok{==} \StringTok{"Silves"}\NormalTok{) }\SpecialCharTok{|\textgreater{}}
  \FunctionTok{select}\NormalTok{(}\SpecialCharTok{{-}}\NormalTok{Concelho)}

\NormalTok{VRSA}\SpecialCharTok{$}\NormalTok{Data }\OtherTok{\textless{}{-}} \FunctionTok{trimws}\NormalTok{(VRSA}\SpecialCharTok{$}\NormalTok{Data)}
\NormalTok{VRSA}\SpecialCharTok{$}\NormalTok{DataDay }\OtherTok{\textless{}{-}} \FunctionTok{paste0}\NormalTok{(VRSA}\SpecialCharTok{$}\NormalTok{Data, }\StringTok{"{-}01"}\NormalTok{)}
\NormalTok{VRSA}\SpecialCharTok{$}\NormalTok{DataDay}\OtherTok{\textless{}{-}} \FunctionTok{as.Date}\NormalTok{(VRSA}\SpecialCharTok{$}\NormalTok{DataDay, }\AttributeTok{format =} \StringTok{"\%Y{-}\%m{-}\%d"}\NormalTok{)}
\NormalTok{VRSA }\OtherTok{\textless{}{-}}\NormalTok{ VRSA }\SpecialCharTok{|\textgreater{}}
  \FunctionTok{select}\NormalTok{(}\SpecialCharTok{{-}}\NormalTok{Data) }\SpecialCharTok{|\textgreater{}}
  \FunctionTok{arrange}\NormalTok{(DataDay) }
\NormalTok{VRSA }\OtherTok{\textless{}{-}}\NormalTok{ VRSA }\SpecialCharTok{|\textgreater{}}
  \FunctionTok{group\_by}\NormalTok{(DataDay) }\SpecialCharTok{|\textgreater{}}
  \FunctionTok{mutate}\NormalTok{(}\AttributeTok{Consumo =} \FunctionTok{mean}\NormalTok{(Energia)) }\SpecialCharTok{|\textgreater{}}
  \FunctionTok{distinct}\NormalTok{(DataDay, Consumo)}

\NormalTok{Bispo}\SpecialCharTok{$}\NormalTok{Data }\OtherTok{\textless{}{-}} \FunctionTok{trimws}\NormalTok{(Bispo}\SpecialCharTok{$}\NormalTok{Data)}
\NormalTok{Bispo}\SpecialCharTok{$}\NormalTok{DataDay }\OtherTok{\textless{}{-}} \FunctionTok{paste0}\NormalTok{(Bispo}\SpecialCharTok{$}\NormalTok{Data, }\StringTok{"{-}01"}\NormalTok{)}
\NormalTok{Bispo}\SpecialCharTok{$}\NormalTok{DataDay}\OtherTok{\textless{}{-}} \FunctionTok{as.Date}\NormalTok{(Bispo}\SpecialCharTok{$}\NormalTok{DataDay, }\AttributeTok{format =} \StringTok{"\%Y{-}\%m{-}\%d"}\NormalTok{)}
\NormalTok{Bispo }\OtherTok{\textless{}{-}}\NormalTok{ Bispo }\SpecialCharTok{|\textgreater{}}
  \FunctionTok{select}\NormalTok{(}\SpecialCharTok{{-}}\NormalTok{Data) }\SpecialCharTok{|\textgreater{}}
  \FunctionTok{arrange}\NormalTok{(DataDay) }
\NormalTok{Bispo }\OtherTok{\textless{}{-}}\NormalTok{ Bispo }\SpecialCharTok{|\textgreater{}}
  \FunctionTok{group\_by}\NormalTok{(DataDay) }\SpecialCharTok{|\textgreater{}}
  \FunctionTok{mutate}\NormalTok{(}\AttributeTok{Consumo =} \FunctionTok{mean}\NormalTok{(Energia)) }\SpecialCharTok{|\textgreater{}}
  \FunctionTok{distinct}\NormalTok{(DataDay, Consumo)}

\NormalTok{Tavira}\SpecialCharTok{$}\NormalTok{Data }\OtherTok{\textless{}{-}} \FunctionTok{trimws}\NormalTok{(Tavira}\SpecialCharTok{$}\NormalTok{Data)}
\NormalTok{Tavira}\SpecialCharTok{$}\NormalTok{DataDay }\OtherTok{\textless{}{-}} \FunctionTok{paste0}\NormalTok{(Tavira}\SpecialCharTok{$}\NormalTok{Data, }\StringTok{"{-}01"}\NormalTok{)}
\NormalTok{Tavira}\SpecialCharTok{$}\NormalTok{DataDay}\OtherTok{\textless{}{-}} \FunctionTok{as.Date}\NormalTok{(Tavira}\SpecialCharTok{$}\NormalTok{DataDay, }\AttributeTok{format =} \StringTok{"\%Y{-}\%m{-}\%d"}\NormalTok{)}
\NormalTok{Tavira }\OtherTok{\textless{}{-}}\NormalTok{ Tavira }\SpecialCharTok{|\textgreater{}}
  \FunctionTok{select}\NormalTok{(}\SpecialCharTok{{-}}\NormalTok{Data) }\SpecialCharTok{|\textgreater{}}
  \FunctionTok{arrange}\NormalTok{(DataDay) }
\NormalTok{Tavira }\OtherTok{\textless{}{-}}\NormalTok{ Tavira }\SpecialCharTok{|\textgreater{}}
  \FunctionTok{group\_by}\NormalTok{(DataDay) }\SpecialCharTok{|\textgreater{}}
  \FunctionTok{mutate}\NormalTok{(}\AttributeTok{Consumo =} \FunctionTok{mean}\NormalTok{(Energia)) }\SpecialCharTok{|\textgreater{}}
  \FunctionTok{distinct}\NormalTok{(DataDay, Consumo)}

\NormalTok{Silves}\SpecialCharTok{$}\NormalTok{Data }\OtherTok{\textless{}{-}} \FunctionTok{trimws}\NormalTok{(Silves}\SpecialCharTok{$}\NormalTok{Data)}
\NormalTok{Silves}\SpecialCharTok{$}\NormalTok{DataDay }\OtherTok{\textless{}{-}} \FunctionTok{paste0}\NormalTok{(Silves}\SpecialCharTok{$}\NormalTok{Data, }\StringTok{"{-}01"}\NormalTok{)}
\NormalTok{Silves}\SpecialCharTok{$}\NormalTok{DataDay}\OtherTok{\textless{}{-}} \FunctionTok{as.Date}\NormalTok{(Silves}\SpecialCharTok{$}\NormalTok{DataDay, }\AttributeTok{format =} \StringTok{"\%Y{-}\%m{-}\%d"}\NormalTok{)}
\NormalTok{Silves }\OtherTok{\textless{}{-}}\NormalTok{ Silves }\SpecialCharTok{|\textgreater{}}
  \FunctionTok{select}\NormalTok{(}\SpecialCharTok{{-}}\NormalTok{Data) }\SpecialCharTok{|\textgreater{}}
  \FunctionTok{arrange}\NormalTok{(DataDay) }
\NormalTok{Silves }\OtherTok{\textless{}{-}}\NormalTok{ Silves }\SpecialCharTok{|\textgreater{}}
  \FunctionTok{group\_by}\NormalTok{(DataDay) }\SpecialCharTok{|\textgreater{}}
  \FunctionTok{mutate}\NormalTok{(}\AttributeTok{Consumo =} \FunctionTok{mean}\NormalTok{(Energia)) }\SpecialCharTok{|\textgreater{}}
  \FunctionTok{distinct}\NormalTok{(DataDay, Consumo)}
\end{Highlighting}
\end{Shaded}

\begin{Shaded}
\begin{Highlighting}[]
\NormalTok{ts.VRSA }\OtherTok{\textless{}{-}} \FunctionTok{ts}\NormalTok{(VRSA}\SpecialCharTok{$}\NormalTok{Consumo, }\AttributeTok{start =} \FunctionTok{c}\NormalTok{(}\DecValTok{2020}\NormalTok{,}\DecValTok{11}\NormalTok{), }\AttributeTok{frequency =} \DecValTok{12}\NormalTok{)}
\NormalTok{ts.Bispo }\OtherTok{\textless{}{-}} \FunctionTok{ts}\NormalTok{(Bispo}\SpecialCharTok{$}\NormalTok{Consumo, }\AttributeTok{start =} \FunctionTok{c}\NormalTok{(}\DecValTok{2020}\NormalTok{,}\DecValTok{11}\NormalTok{), }\AttributeTok{frequency =} \DecValTok{12}\NormalTok{)}
\NormalTok{ts.Tavira }\OtherTok{\textless{}{-}} \FunctionTok{ts}\NormalTok{(Tavira}\SpecialCharTok{$}\NormalTok{Consumo, }\AttributeTok{start =} \FunctionTok{c}\NormalTok{(}\DecValTok{2020}\NormalTok{,}\DecValTok{11}\NormalTok{), }\AttributeTok{frequency =} \DecValTok{12}\NormalTok{)}
\NormalTok{ts.Silves }\OtherTok{\textless{}{-}} \FunctionTok{ts}\NormalTok{(Silves}\SpecialCharTok{$}\NormalTok{Consumo, }\AttributeTok{start =} \FunctionTok{c}\NormalTok{(}\DecValTok{2020}\NormalTok{,}\DecValTok{11}\NormalTok{), }\AttributeTok{frequency =} \DecValTok{12}\NormalTok{)}
\end{Highlighting}
\end{Shaded}

\subsection{Questão 1}\label{questuxe3o-1}

\begin{Shaded}
\begin{Highlighting}[]
\FunctionTok{autoplot}\NormalTok{(ts.VRSA, }\AttributeTok{xlab =} \StringTok{"Tempo"}\NormalTok{, }\AttributeTok{ylab =} \StringTok{"Consumo de Energia"}\NormalTok{, }\AttributeTok{main =} \StringTok{"Consumo de Energia em Vila Real de Santo Antônio"}\NormalTok{)}
\end{Highlighting}
\end{Shaded}

\includegraphics{Carmelo_files/figure-latex/unnamed-chunk-3-1.pdf}

Com relação ao consumo de energia de Vila Real de Santo Antônio podemos
observar uma sazonalidade muito proeminte. Com relação a tendência,
podemos observar um leve, mas muito leve, crescimento anual.

\begin{Shaded}
\begin{Highlighting}[]
\FunctionTok{autoplot}\NormalTok{(ts.Bispo, }\AttributeTok{xlab =} \StringTok{"Tempo"}\NormalTok{, }\AttributeTok{ylab =} \StringTok{"Consumo de Energia"}\NormalTok{, }\AttributeTok{main =} \StringTok{"Consumo de Energia em Vila do Bispo"}\NormalTok{)}
\end{Highlighting}
\end{Shaded}

\includegraphics{Carmelo_files/figure-latex/unnamed-chunk-4-1.pdf}

Comentando sobre Vila do Bispo podemos observar que não há uma
sazonalidade tão bem demarcada, muito menos uma tendência forte. Quando
observamos podemos ver que a série sofre de uma grande variação nos seus
dados ao longo do período.

\begin{Shaded}
\begin{Highlighting}[]
\FunctionTok{autoplot}\NormalTok{(ts.Tavira, }\AttributeTok{xlab =} \StringTok{"Tempo"}\NormalTok{, }\AttributeTok{ylab =} \StringTok{"Consumo de Energia"}\NormalTok{, }\AttributeTok{main =} \StringTok{"Consumo de Energia em Tavira"}\NormalTok{)}
\end{Highlighting}
\end{Shaded}

\includegraphics{Carmelo_files/figure-latex/unnamed-chunk-5-1.pdf}

Tavira por outro lado possui sim um componente sazonal bastante visível
aos olhos, variando ao longo dos anos a amplitude entre os dados.
Outrossim, não podemos dizer que há uma tendência demarcada.

\begin{Shaded}
\begin{Highlighting}[]
\FunctionTok{autoplot}\NormalTok{(ts.Silves, }\AttributeTok{xlab =} \StringTok{"Tempo"}\NormalTok{, }\AttributeTok{ylab =} \StringTok{"Consumo de Energia"}\NormalTok{, }\AttributeTok{main =} \StringTok{"Consumo de Energia em Silves"}\NormalTok{)}
\end{Highlighting}
\end{Shaded}

\includegraphics{Carmelo_files/figure-latex/unnamed-chunk-6-1.pdf}

Por último, mas não menos importante podemos observar Silves, que assim
como Tavria tem uma sazonalidade bastante demarcada em primeira vista.
Porém ao contrário de sua vizinha, Silves apresenta uma leve tendência
positiva ao longo dos anos.

\subsection{Questão 2}\label{questuxe3o-2}

\begin{Shaded}
\begin{Highlighting}[]
\FunctionTok{ggseasonplot}\NormalTok{(ts.VRSA, }\AttributeTok{main =} \StringTok{"Consumo Sazonal de Energia em Vila Real de Santo Antônio"}\NormalTok{, }\AttributeTok{xlab =} \StringTok{"Mês"}\NormalTok{)}
\end{Highlighting}
\end{Shaded}

\includegraphics{Carmelo_files/figure-latex/unnamed-chunk-7-1.pdf}

Com execessão de 2021, em Vila Real de Santo Antônio, podemos observar
uma sobreposição dos dados quase perfeita dos dados, o que denota um
componente sazonal bastante forte. Mas vale ressaltar o mês de agosto,
que representa sucessivos aumentos ao longo de 2020 e 2024 (Mais uma vez
excluindo 2021).

\begin{Shaded}
\begin{Highlighting}[]
\FunctionTok{ggseasonplot}\NormalTok{(ts.Bispo, }\AttributeTok{main =} \StringTok{"Consumo Sazonal de Energia em Vila do Bispo"}\NormalTok{, }\AttributeTok{xlab =} \StringTok{"Mês"}\NormalTok{)}
\end{Highlighting}
\end{Shaded}

\includegraphics{Carmelo_files/figure-latex/unnamed-chunk-8-1.pdf}

Tal qual Vila Real, a Vila do Bispo também apresenta em 2021 um
comportamento quase que anômalo com relação ao restante dos anos na
série, em especial no começo do ano, onde a diferença é bastante
notória. Com relação aos outros meses também podemos observar uma
sazonalidade mais fraca em relação a anterior e com uma leve queda de
consumo nos últimos anos.

\begin{Shaded}
\begin{Highlighting}[]
\FunctionTok{ggseasonplot}\NormalTok{(ts.Tavira, }\AttributeTok{main =} \StringTok{"Consumo Sazonal de Energia em Tavira"}\NormalTok{, }\AttributeTok{xlab =} \StringTok{"Mês"}\NormalTok{)}
\end{Highlighting}
\end{Shaded}

\includegraphics{Carmelo_files/figure-latex/unnamed-chunk-9-1.pdf}

Em Tavira podemos observar que com os consumos mensais, em especial nos
meses pós agosto são muito parecidos, e parecem lentamente convergir
mais com o passar do tempo, dado a proximidade dos dados pós 2021.

\begin{Shaded}
\begin{Highlighting}[]
\FunctionTok{ggseasonplot}\NormalTok{(ts.Silves, }\AttributeTok{main =} \StringTok{"Consumo Sazonal de Energia em Silves"}\NormalTok{, }\AttributeTok{xlab =} \StringTok{"Mês"}\NormalTok{)}
\end{Highlighting}
\end{Shaded}

\includegraphics{Carmelo_files/figure-latex/unnamed-chunk-10-1.pdf}

Em Silves, tal qual a maioria dos Demais podemos observar de forma clara
como a sazonalidade é bastante demarcada, em especial após 2021.

\subsection{Questão 3}\label{questuxe3o-3}

\begin{Shaded}
\begin{Highlighting}[]
\NormalTok{VRSA.decA }\OtherTok{\textless{}{-}} \FunctionTok{decompose}\NormalTok{(ts.VRSA, }\AttributeTok{type =} \StringTok{"additive"}\NormalTok{)}
\NormalTok{VRSA.decM }\OtherTok{\textless{}{-}} \FunctionTok{decompose}\NormalTok{(ts.VRSA, }\AttributeTok{type =} \StringTok{"multi"}\NormalTok{)}
\end{Highlighting}
\end{Shaded}

\begin{Shaded}
\begin{Highlighting}[]
\FunctionTok{autoplot}\NormalTok{(VRSA.decA)}
\end{Highlighting}
\end{Shaded}

\includegraphics{Carmelo_files/figure-latex/unnamed-chunk-12-1.pdf}

\begin{Shaded}
\begin{Highlighting}[]
\FunctionTok{autoplot}\NormalTok{(VRSA.decM)}
\end{Highlighting}
\end{Shaded}

\includegraphics{Carmelo_files/figure-latex/unnamed-chunk-13-1.pdf}

\begin{Shaded}
\begin{Highlighting}[]
\NormalTok{Bispo.decA }\OtherTok{\textless{}{-}} \FunctionTok{decompose}\NormalTok{(ts.Bispo, }\AttributeTok{type =} \StringTok{"additive"}\NormalTok{)}
\NormalTok{Bispo.decM }\OtherTok{\textless{}{-}} \FunctionTok{decompose}\NormalTok{(ts.Bispo, }\AttributeTok{type =} \StringTok{"multi"}\NormalTok{)}
\end{Highlighting}
\end{Shaded}

\begin{Shaded}
\begin{Highlighting}[]
\FunctionTok{autoplot}\NormalTok{(Bispo.decA)}
\end{Highlighting}
\end{Shaded}

\includegraphics{Carmelo_files/figure-latex/unnamed-chunk-15-1.pdf}

\begin{Shaded}
\begin{Highlighting}[]
\FunctionTok{autoplot}\NormalTok{(Bispo.decM)}
\end{Highlighting}
\end{Shaded}

\includegraphics{Carmelo_files/figure-latex/unnamed-chunk-16-1.pdf}

\begin{Shaded}
\begin{Highlighting}[]
\NormalTok{Tavira.decA }\OtherTok{\textless{}{-}} \FunctionTok{decompose}\NormalTok{(ts.Tavira, }\AttributeTok{type =} \StringTok{"additive"}\NormalTok{)}
\NormalTok{Tavira.decM }\OtherTok{\textless{}{-}} \FunctionTok{decompose}\NormalTok{(ts.Tavira, }\AttributeTok{type =} \StringTok{"multi"}\NormalTok{)}
\end{Highlighting}
\end{Shaded}

\begin{Shaded}
\begin{Highlighting}[]
\FunctionTok{autoplot}\NormalTok{(Tavira.decA)}
\end{Highlighting}
\end{Shaded}

\includegraphics{Carmelo_files/figure-latex/unnamed-chunk-18-1.pdf}

\begin{Shaded}
\begin{Highlighting}[]
\FunctionTok{autoplot}\NormalTok{(Tavira.decM)}
\end{Highlighting}
\end{Shaded}

\includegraphics{Carmelo_files/figure-latex/unnamed-chunk-19-1.pdf}

\begin{Shaded}
\begin{Highlighting}[]
\NormalTok{Silves.decA }\OtherTok{\textless{}{-}} \FunctionTok{decompose}\NormalTok{(ts.Silves, }\AttributeTok{type =} \StringTok{"additive"}\NormalTok{)}
\NormalTok{Silves.decM }\OtherTok{\textless{}{-}} \FunctionTok{decompose}\NormalTok{(ts.Silves, }\AttributeTok{type =} \StringTok{"multi"}\NormalTok{)}
\end{Highlighting}
\end{Shaded}

\begin{Shaded}
\begin{Highlighting}[]
\FunctionTok{autoplot}\NormalTok{(Silves.decA)}
\end{Highlighting}
\end{Shaded}

\includegraphics{Carmelo_files/figure-latex/unnamed-chunk-21-1.pdf}

\begin{Shaded}
\begin{Highlighting}[]
\FunctionTok{autoplot}\NormalTok{(Silves.decM)}
\end{Highlighting}
\end{Shaded}

\includegraphics{Carmelo_files/figure-latex/unnamed-chunk-22-1.pdf}

\end{document}
